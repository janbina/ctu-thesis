%! Suppress = PrimitiveStyle
%! Suppress = PrimitiveEquation
\label[chap:icccmewmh]

% TODO: what about this???
%\chap ICCCM and EWMH
%It was an explicit design goal of X Version 11 to specify mechanism, not policy.
%Because of that, a client that communicates with the server using the protocol defined by
%the X Window System Protocol (discussed in Chapter~\ref[chap:xsystem])
%may operate correctly in isolation but may not coexist properly with others sharing the same server.~\cite[icccm]
%
%Standardized communication is important especially for window managers and other special clients like docks, toolbars or pagers,
%because they need to communicate with the rest of the clients as well as with each other.
%In this chapter, we will discuss two such standard protocols defined on top of the X protocol:
%\begitems
%* ICCCM -- Inter-Client Communication Conventions Manual
%* EWMH -- Extended Window Manager Hints
%\enditems

\label[sec:icccm]
\sec ICCCM
It was an explicit design goal of X Version 11 to specify a mechanism, not a policy.
Because of that, a client that communicates with an X11 server using the protocol defined by
the X Window System Protocol (discussed in Section~\ref[sec:xsystem])
may operate correctly in isolation but may not coexist properly with others sharing the same server~\cite[icccm].

Standardized communication is important especially for window managers and other special clients like docks, toolbars or pagers,
because they need to communicate with the rest of the clients as well as with each other.
Inter-Client Communication Conventions Manual is one of two standards that define how X Window System clients should interact with one another.
It was designed at the X Consortium in 1988.
Its latest version 2.0 was released in 1994~\cite[icccm].

In this section, we will cover parts of ICCCM that are related to window managers and therefore important for us.
ICCCM manual~\cite[icccm] will be constantly referenced here, so we will explicitly refer only other sources if they are used.

%The ICCCM is known for being difficult to correctly implement.
%Moreover, some of its parts are obsolete or no longer practical to implement.
%Efforts to extend the ICCCM for current needs have resulted in creation of the Extended Window Manager Hints, which will be discussed in next section.
%
%To permit window managers to perform their role of mediating the competing demands for resources such as screen space, the clients being managed must adhere to certain conventions and must expect the window managers to do likewise. These conventions are covered here from the client's point of view.
%
%In general, these conventions are somewhat complex and will undoubtedly change as new window management paradigms are developed. Thus, there is a strong bias toward defining only those conventions that are essential and that apply generally to all window management paradigms. Clients designed to run with a particular window manager can easily define private protocols to add to these conventions, but they must be aware that their users may decide to run some other window manager no matter how much the designers of the private protocol are convinced that they have seen the "one true light" of user interfaces.
%
%It is a principle of these conventions that a general client should neither know nor care which window manager is running or, indeed, if one is running at all. The conventions do not support all client functions without a window manager running; for example, the concept of Iconic is not directly supported by clients. If no window manager is running, the concept of Iconic does not apply. A goal of the conventions is to make it possible to kill and restart window managers without loss of functionality.
%
%Each window manager will implement a particular window management policy; the choice of an appropriate window management policy for the user's circumstances is not one for an individual client to make but will be made by the user or the user's system administrator. This does not exclude the possibility of writing clients that use a private protocol to restrict themselves to operating only under a specific window manager. Rather, it merely ensures that no claim of general utility is made for such programs.
%
%For example, the claim is often made: "The client I'm writing is important, and it needs to be on top." Perhaps it is important when it is being run in earnest, and it should then be run under the control of a window manager that recognizes "important" windows through some private protocol and ensures that they are on top. However, imagine, for example, that the "important" client is being debugged. Then, ensuring that it is always on top is no longer the appropriate window management policy, and it should be run under a window manager that allows other windows (for example, the debugger) to appear on top.

\label[sec:icccmselection]
\secc Selection
Selections are the primary mechanism that X11 defines for the exchange of information between clients.
There can be an arbitrary number of selections, each of them is named by an atom, and they are global to the server.
Each selection can be owned by some client and attached to a window created by that client.

For our window manager use case, we will be concerned about ownership of a selection named "WM_Sn", where "n" is the screen number.
To do its job, the window manager needs to register for "SubstructureRedirect" events on the root window of the screen it wants to manage
(we will discuss this in more detail in Section~\ref[sec:becomingwm]).
Since only one client can be registered for substructure redirection on any given window at any given time,
we need some mechanism to inform the client that is currently registered for it that we (on behalf of the user)
want to replace it.
Selection provides this mechanism.

To see this in practice, look at Listing~\ref[code:selection]
(code is in Go, some of its constructs are explained in a comment where needed).
Firstly, on line 2, we retrieve the owner of the selection we want to manage ("WM_S2").
We then check if another client owns this selection, in which case we should get confirmation
from the user that our application could replace it.
If there is no owner or we have the user's permission to replace it,
we send the "SetSelectionOwner" request, specifying the selection atom ("WM_S2") and the window to which the selection
would be attached ("X.Dummy()" in this case).
If the selection was owned by another client, we then have to wait for its termination, that is, wait for
the "DestroyNotifyEvent" of the window that owned the selection.
After that, we can finally register for "SubstructureRedirect" events on the root window of screen 2, since
we now own the selection "WM_S2".
The last thing we have to do to comply with ICCCM is to listen for the "SelectionClearEvent".
We will receive this event when another client sends the "SetSelectionOwner" request (just like we did earlier)
and we have to release managed resources (substructure redirection on root window) and destroy the window that owns
the selection.
In our case, we simply quit, as there is no point for the window manager to run without the substructure redirection.

\midinsert
\hisyntax{GO}
\verbinput (-) listings/selection.go
\clabel[code:selection]{Selection acquiring mechanism}
\caption/l {
    Selection acquiring mechanism (code is simplified and would not compile as is)
}
\endinsert



%As stated by the ICCCM, window managers should acquire ownership of a selection named "WM_Sn", where "n" is the screen number, for every screen they manage.
%Selection ownership is acquired by calling "SetSelectionOwner" request, specifying the atom that represents the selection, some window that the client created
%and timestamp of the event that triggers the acquisition of the selection.
%If the time in the "SetSelectionOwner" request is in the future relative to the server's current time or is in the past relative
%to the last time the specified selection changed hands, the "SetSelectionOwner" request appears to the client to succeed, but ownership is not actually transferred.
%Client can get the owner of a particular selection by invoking "GetSelectionOwner" request.
%
%Clients may either give up selection ownership voluntarily or lose it forcibly as the result of some other client's actions.
%To relinquish ownership of a selection voluntarily, a client should execute a "SetSelectionOwner" request for that selection atom,
%with owner specified as "None" and the time specified as the timestamp that was used to acquire the selection.
%Alternatively, the client may destroy the window used as the owner value of the "SetSelectionOwner" request, or the client may terminate.
%In both cases, the ownership of the selection involved will revert to "None".
%If a client gives up ownership of a selection or if some other client executes a "SetSelectionOwner" for it and thus reassigns it forcibly,
%the previous owner will receive a "SelectionClear" event.
%
%If a manager loses ownership of a manager selection, a new manager is taking over its responsibilities.
%The old manager must release all resources it has managed and must then destroy the window that owned the selection.
%For example, a window manager losing ownership of "WM_S2" must deselect from "SubstructureRedirect" on the root window of screen 2 before destroying the window that owned "WM_S2".
%When the new manager notices that the window owning the selection has been destroyed,
%it knows that it can successfully proceed to control the resource it is planning to manage.
%If the old manager does not destroy the window within a reasonable time, the new manager should check with the user before destroying the window itself or killing the old manager.

\secc Clients Actions
Clients should do exactly what they would do in the absence of a window manager, with following exceptions:
\begitems
* They should hint to the window manager what resources they would like to obtain.
* They should accept the resources they are allocated, even if they are not those requested.
* They should be prepared for resource allocations to change at any time.
\enditems

\label[sec:toplevelwindow]
\secc Creating a Top-Level Window
Top-Level window is a window whose override-redirect attribute is "false".
It must either be a child of a root window, or it must have been a child of a root window immediately prior to having been reparented by the window manager.
From the client's point of view, the window manager will regard its top-level window as being in one of three states:
\begitems
* Normal
* Iconic
* Withdrawn
\enditems
Newly created windows start in the Withdrawn state.
Transitions between states happen when the top-level window is {\em mapped} and {\em unmapped} and when the window manager receives certain messages.
For historical reasons related to some initial implementations, showing a window in the X11 protocol is called mapping a window,
and hiding a window is called unmapping~\cite[xBasics].
Transitions between those states will be described in Section~\ref[sec:changingstate].

\label[sec:clientproperties]
\secc Client Properties
Client can inform the window manager of the behavior that it desires by placing properties on its top-level windows.
Window manager is free to assume values it finds convenient for any properties that are not supplied.
The window manager will not change properties written by the client.
Contents of these properties are examined by the window manager upon transition from the "Withdrawn" state,
some properties are also monitored for changes while the window is in the "Iconic" or "Normal" state.
For example, a file manager usually changes "WM_NAME" property when the user navigates to different directories,
and the window manager is expected to observe these changes and reflect them in its UI.

In Table~\ref[tab:propsummary], we can see a summary of all client properties defined by ICCCM.
Some of them will be covered in detail.

\midinsert \clabel[tab:propsummary]{Summary of Window Manager Property Types}
\ctable{llr}{
    Name & Type\crl\tskip4pt
    "WM\_CLASS" & "STRING" \cr
    "WM\_CLIENT\_MACHINE" & "TEXT" \cr
    "WM\_COLORMAP\_WINDOWS" & "WINDOW" \cr
    "WM\_HINTS" & "WM\_HINTS" \cr
    "WM\_ICON\_NAME" & "TEXT" \cr
    "WM\_ICON\_SIZE" & "WM\_ICON\_SIZE" \cr
    "WM\_NAME" & "TEXT" \cr
    "WM\_NORMAL\_HINTS" & "WM\_SIZE\_HINTS" \cr
    "WM\_PROTOCOLS" & "ATOM" \cr
    "WM\_STATE" & "WM\_STATE" \cr
    "WM\_TRANSIENT\_FOR" & "WINDOW" \cr
}
\caption/t Summary of window manager property types
\endinsert

\heading WM\uns CLASS
The "WM_CLASS" property contains two consecutive null-terminated strings.
These specify the Instance and Class names to be used by both the client and the window manager for looking up resources for the application or as an identifying information.
For example, in Table~\ref[tab:properties], we have seen a window with "WM_CLASS" set to {\em ``jetbrains-goland, jetbrains-goland''}.

\heading{WM\uns NAME, WM\uns ICON\uns NAME}
The "WM_NAME" property is an uninterpreted string that the client wants the window manager
to display in association with the window (for example, in a window title bar).
Window managers are expected to make an effort to display this information, ignoring "WM_NAME" is not acceptable behavior,
unless the user has taken an explicit action to make it invisible.
The "WM_ICON_NAME" property is similar, but it is used when the window is in "Iconic" state.

\heading WM\uns NORMAL\uns HINTS
The "WM_NORMAL_HINTS" property is used by clients to specify the minimum size that the window can be for the client to be useful,
as well as the maximum size.
Window managers should honor them, even though they do not have to.
Clients can also specify base size and size increments.
If they are specified, the window manager should not just resize the window to an arbitrary size,
but the size should reflect those values in this way: $size = base\_size + i * size\_increment$.
This is used, for example, by terminal emulators that want to fit an exact number of characters into the window.

\heading WM\uns PROTOCOLS
The "WM_PROTOCOLS" property is a list of atoms.
Each atom identifies a communication protocol between the client and the window manager in which the client is willing to participate.
Atoms can identify both standard protocols and private protocols specific to individual window managers.
There are three protocols defined by the ICCCM at the moment:
\begitems
* "WM_TAKE_FOCUS" -- assignment of input focus.
* "WM_DELETE_WINDOW" -- request to delete top level window.
* "WM_SAVE_YOURSELF" -- request to save client state (deprecated).
\enditems

\heading WM\uns STATE
The "WM_STATE" property is placed on each top-level client window that is not in the "Withdrawn" state by the window manager.
It specifies the state of the window ("Withdrawn", "Normal", or "Iconic"),
and ID of icon window (if the window is in "Iconic" state and the window manager is displaying some).

The "WM_STATE" property is often used as an indicator of a top-level window.
For example, some clients (such as xprop~\cite[xprop])
need to find a top-level window under the pointer (user clicking on a window).
They can do it by searching the window hierarchy beneath the selected location for a window with the "WM_STATE" property.

\heading WM\uns TRANSIENT\uns FOR
The "WM_TRANSIENT_FOR" property might contain ID of another top-level window.
The implication is that this window is a pop-up on behalf of the named window, and window managers may decide
not to decorate transient windows or may treat them differently in other ways.
In particular, window managers should present newly mapped "WM_TRANSIENT_FOR" windows without requiring any user interaction,
even if mapping top-level windows normally does require interaction.
Dialogs, for example, are an example of windows that should have "WM_TRANSIENT_FOR" set.

\heading WM\uns HINTS
The "WM_HINTS" property is used to communicate information that does not need separate properties to the window manager.
For our use case, particularly useful fields would be "flags", "initial_state", and "window_group".
Field "flags" contains boolean value for "UrgencyHint", which is set to "true" when the window needs user attention.
Window manager should communicate this to the user -- for example, change color of window's border.
The value of the "initial_state" field determines the state the client wishes to be in when the top-level window is firstly mapped.
It could be either "NormalState" (window is visible) or "IconicState" (icon is visible) -- see Section~\ref[sec:toplevelwindow].
The "window_group" field lets the client specify that this window belongs to a group of windows.
Window manager can use this hint to manipulate the group as a whole.

\label[sec:changingstate]
\secc Changing Window State
As already mentioned in Section~\ref[sec:toplevelwindow], window manager will assign each top-level window one of three states:
\begitems
* "Normal" -- top-level window is visible.
* "Iconic" -- top-level window is iconic.
That usually means that top-level window is not visible, but its "icon_window", "icon_pixmap" or "WM_ICON_NAME" is.
* "Withdrawn" -- neither top-level window nor icon is visible.
\enditems
Newly created top-level windows are in the "Withdrawn" state.
Once the window has been provided with suitable properties, the client is free to change its state as follows:
\begitems
* "Withdrawn"\,\to\,"Normal" -- when window is mapped with "WM_HINTS.initial_state" being "NormalState".
* "Withdrawn"\,\to\,"Iconic" -- when window is mapped with "WM_HINTS.initial_state" being "IconicState".
* "Normal"\,\to\,"Iconic" -- clint sends "ClientMessage" event.
* "Normal"\,\to\,"Withdrawn" -- client unmaps window and sends a synthetic "UnmapNotify" event.
* "Iconic"\,\to\,"Normal" -- client maps the window, content of "WM_HINTS.initial_state" is irrelevant in this case.
* "Iconic"\,\to\,"Withdrawn" -- client unmaps window and sends a synthetic "UnmapNotify" event.
\enditems

\secc Configuring the Window
Clients can resize and reposition their top-level windows by using the "ConfigureWindow" request.
The attributes of the window that can be altered with this request are as follows:
\begitems
* The $[x,y]$ location of the window's upper left-outer corner.
* The width and height of the inner region of the window (excluding borders).
* The width of the border of the window.
* The window's position in the stack.
\enditems
The coordinate system in which the location is expressed is that of the root (irrespective of any reparenting that may have occurred).
Client configure requests are interpreted by the window manager in the same manner as the initial window geometry specified by the "WM_NORMAL_HINTS" property.
Clients must be aware that there is no guarantee that the window manager will allocate them the requested size or location
and must be prepared to deal with any size and location.
Window manager can respond to a "ConfigureRequest" request in three different ways:
\begitems
* Not change the size, location, border width, or stacking order of the window at all.\nl
A~client will receive a synthetic "ConfigureNotify" event that describes the (unchanged) geometry of the window.
The client will not receive a real "ConfigureNotify" event because no change has actually taken place.
* Move or restack the window without resizing it or changing its border width.\nl
A~client will receive a synthetic "ConfigureNotify" event following the change that describes the new geometry of the window.
* Resize the window or change its border width.\nl
A~client will receive a real "ConfigureNotify" event, provided that it has selected for "StructureNotify" events.
\enditems
