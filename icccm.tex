%! Suppress = PrimitiveStyle
%! Suppress = PrimitiveEquation
\label[chap:icccmewmh]

% TODO: what about this???
%\chap ICCCM and EWMH
%It was an explicit design goal of X Version 11 to specify mechanism, not policy.
%Because of that, a client that communicates with the server using the protocol defined by
%the X Window System Protocol (discussed in chapter~\ref[chap:xsystem])
%may operate correctly in isolation but may not coexist properly with others sharing the same server.~\cite[icccm]
%
%Standardized communication is important especially for window managers and other special clients like docks, toolbars or pagers,
%because they need to communicate with the rest of the clients as well as with each other.
%In this chapter, we will discuss two such standard protocols defined on top of the X protocol:
%\begitems
%* ICCCM -- Inter-Client Communication Conventions Manual
%* EWMH -- Extended Window Manager Hints
%\enditems

\label[sec:icccm]
\sec ICCCM
It was an explicit design goal of X Version 11 to specify a mechanism, not a policy.
Because of that, a client that communicates with an X11 server using the protocol defined by
the X Window System Protocol (discussed in previous section~\ref[sec:xsystem])
may operate correctly in isolation but may not coexist properly with others sharing the same server.~\cite[icccm]

Standardized communication is important especially for window managers and other special clients like docks, toolbars or pagers,
because they need to communicate with the rest of the clients as well as with each other.
Inter-Client Communication Conventions Manual is one of two standards that define how X Window System clients should interact with one another.
It was designed at the X Consortium in 1988.
Its latest version 2.0 was released in 1994.~\cite[icccm]

In this section, we will cover parts of ICCCM that are related to window managers and therefore important for us.
ICCCM manual~\cite[icccm] will be constantly referenced here, so we will explicitly refer only other sources if they are used.

%The ICCCM is known for being difficult to correctly implement.
%Moreover, some of its parts are obsolete or no longer practical to implement.
%Efforts to extend the ICCCM for current needs have resulted in creation of the Extended Window Manager Hints, which will be discussed in next section.
%
%To permit window managers to perform their role of mediating the competing demands for resources such as screen space, the clients being managed must adhere to certain conventions and must expect the window managers to do likewise. These conventions are covered here from the client's point of view.
%
%In general, these conventions are somewhat complex and will undoubtedly change as new window management paradigms are developed. Thus, there is a strong bias toward defining only those conventions that are essential and that apply generally to all window management paradigms. Clients designed to run with a particular window manager can easily define private protocols to add to these conventions, but they must be aware that their users may decide to run some other window manager no matter how much the designers of the private protocol are convinced that they have seen the "one true light" of user interfaces.
%
%It is a principle of these conventions that a general client should neither know nor care which window manager is running or, indeed, if one is running at all. The conventions do not support all client functions without a window manager running; for example, the concept of Iconic is not directly supported by clients. If no window manager is running, the concept of Iconic does not apply. A goal of the conventions is to make it possible to kill and restart window managers without loss of functionality.
%
%Each window manager will implement a particular window management policy; the choice of an appropriate window management policy for the user's circumstances is not one for an individual client to make but will be made by the user or the user's system administrator. This does not exclude the possibility of writing clients that use a private protocol to restrict themselves to operating only under a specific window manager. Rather, it merely ensures that no claim of general utility is made for such programs.
%
%For example, the claim is often made: "The client I'm writing is important, and it needs to be on top." Perhaps it is important when it is being run in earnest, and it should then be run under the control of a window manager that recognizes "important" windows through some private protocol and ensures that they are on top. However, imagine, for example, that the "important" client is being debugged. Then, ensuring that it is always on top is no longer the appropriate window management policy, and it should be run under a window manager that allows other windows (for example, the debugger) to appear on top.

\label[sec:icccmselection]
\secc Selection
Selections are the primary mechanism that X11 defines for the exchange of information between clients.
There can be an arbitrary number of selections, each of them is named by an atom, and they are global to the server.
Each selection can be owned by some client and attached to a window created by that client.

For our window manager use case, we will be concerned about ownership of a selection named "WM_Sn", where "n" is the screen number.
To do its job, the window manager needs to register for "SubstructureRedirect" events on the root window of the screen it wants to manage
(we will discuss this in more detail in section~\ref[sec:becomingwm]).
Since only one client can be registered for substructure redirection on any given window at any given time,
we need some mechanism to inform the client that is currently registered for it that we (on behalf of the user)
want to replace it.
Selection provides this mechanism.

To see this in practice, look at code listing~\ref[code:selection].
Firstly, on line 2, we retrieve the owner of the selection we want to manage ("WM_S2").
We then check if another client owns this selection, in which case we should get confirmation
from the user that our application could replace it.
If there is no owner or we have the user's permission to replace it,
we send the "SetSelectionOwner" request, specifying the selection atom ("WM_S2") and the window to which the selection
would be attached ("X.Dummy()" in this case).
If the selection was owned by another client, we then have to wait for its termination, that is, wait for
the "DestroyNotifyEvent" of the window that owned the selection.
After that, we can finally register for "SubstructureRedirect" events on the root window of screen 2, since
we now own the selection "WM_S2".
The last thing we have to do to comply with ICCCM is to listen for the "SelectionClearEvent".
We will receive this event when another client sends the "SetSelectionOwner" request (just like we did earlier)
and we have to release managed resources (substructure redirection on root window) and destroy the window that owns
the selection.
In our case, we simply quit, as there is no point for the window manager to run without the substructure redirection.

\midinsert
\verbinput (-) listings/selection.go
\clabel[code:selection]{Selection acquiring mechanism}
\caption/l {
    Selection acquiring mechanism.
    Note that the code is simplified a lot and would not compile like this.
}
\endinsert



%As stated by the ICCCM, window managers should acquire ownership of a selection named "WM_Sn", where "n" is the screen number, for every screen they manage.
%Selection ownership is acquired by calling "SetSelectionOwner" request, specifying the atom that represents the selection, some window that the client created
%and timestamp of the event that triggers the acquisition of the selection.
%If the time in the "SetSelectionOwner" request is in the future relative to the server's current time or is in the past relative
%to the last time the specified selection changed hands, the "SetSelectionOwner" request appears to the client to succeed, but ownership is not actually transferred.
%Client can get the owner of a particular selection by invoking "GetSelectionOwner" request.
%
%Clients may either give up selection ownership voluntarily or lose it forcibly as the result of some other client's actions.
%To relinquish ownership of a selection voluntarily, a client should execute a "SetSelectionOwner" request for that selection atom,
%with owner specified as "None" and the time specified as the timestamp that was used to acquire the selection.
%Alternatively, the client may destroy the window used as the owner value of the "SetSelectionOwner" request, or the client may terminate.
%In both cases, the ownership of the selection involved will revert to "None".
%If a client gives up ownership of a selection or if some other client executes a "SetSelectionOwner" for it and thus reassigns it forcibly,
%the previous owner will receive a "SelectionClear" event.
%
%If a manager loses ownership of a manager selection, a new manager is taking over its responsibilities.
%The old manager must release all resources it has managed and must then destroy the window that owned the selection.
%For example, a window manager losing ownership of "WM_S2" must deselect from "SubstructureRedirect" on the root window of screen 2 before destroying the window that owned "WM_S2".
%When the new manager notices that the window owning the selection has been destroyed,
%it knows that it can successfully proceed to control the resource it is planning to manage.
%If the old manager does not destroy the window within a reasonable time, the new manager should check with the user before destroying the window itself or killing the old manager.

\secc Clients Actions
Clients should, as far as possible, do exactly what they would do in the absence of a window manager,
except for the following:
\begitems
* Hinting to the window manager about the resources they would like to obtain.
* Cooperating with the window manager by accepting the resources they are allocated even if they are not those requested.
* Being prepared for resource allocations to change at any time.
\enditems

\label[sec:toplevelwindow]
\secc Creating a Top-Level Window
Top-Level window is a window whose override-redirect attribute is "false".
It must either be a child of a root window, or it must have been a child of a root window immediately prior to having been reparented by the window manager.
From the client's point of view, the window manager will regard its top-level window as being in one of three states:
\begitems
* Normal
* Iconic
* Withdrawn
\enditems
Newly created windows start in the Withdrawn state.
Transitions between states happen when the top-level window is {\em mapped} and {\em unmapped} and when the window manager receives certain messages.
For historical reasons related to some initial implementations, showing a window in the X11 protocol is called mapping a window,
and hiding a window is called unmapping~\cite[xBasics].
Transitions between those states will be described in section~\ref[sec:changingstate].

\label[sec:clientproperties]
\secc Client Properties
Once the client has one or more top-level windows, it should place properties on those windows to inform the window manager of the behavior that the client desires.
If any of these properties are not supplied, window managers can assume values they find convenient -- clients that depend on particular values must explicitly supply them.
The window manager will not change properties written by the client.

By convention, clients writing or rewriting window manager properties must ensure that the entire content of each property remains valid at all times.
Also, all fields of the property must be set to suitable values in a single "ChangeProperty" request.
This ensures that the full contents of the property will be available to a new window manager if the existing one crashes or shuts down.
Contents of these properties are examined by the window manager upon transition from the "Withdrawn" state,
some properties are also monitored for changes while the window is in the "Iconic" or "Normal" state.
For example, a file manager usually changes "WM\_NAME" property when the user navigates to different directories,
and the window manager is expected to observe these changes and reflect them in its UI.

Extending these properties is reserved to the X Consortium; private extensions to them are forbidden.
Private additional communication between clients and window managers should take place using separate properties.
The only exception to this rule is the "WM_PROTOCOLS" property, which may be of arbitrary length and which may contain atoms representing private protocols.

In table~\ref[tab:propsummary], we can see a summary of all client properties defined by ICCCM.
Some of them will be covered in detail.

\midinsert \clabel[tab:propsummary]{Summary of Window Manager Property Types}
\ctable{llr}{
    Name & Type\crl\tskip4pt
    "WM\_CLASS" & "STRING" \cr
    "WM\_CLIENT\_MACHINE" & "TEXT" \cr
    "WM\_COLORMAP\_WINDOWS" & "WINDOW" \cr
    "WM\_HINTS" & "WM\_HINTS" \cr
    "WM\_ICON\_NAME" & "TEXT" \cr
    "WM\_ICON\_SIZE" & "WM\_ICON\_SIZE" \cr
    "WM\_NAME" & "TEXT" \cr
    "WM\_NORMAL\_HINTS" & "WM\_SIZE\_HINTS" \cr
    "WM\_PROTOCOLS" & "ATOM" \cr
    "WM\_STATE" & "WM\_STATE" \cr
    "WM\_TRANSIENT\_FOR" & "WINDOW" \cr
}
\caption/t Summary of Window Manager Property Types.
\endinsert

\heading WM\uns CLASS
The "WM_CLASS" property contains two consecutive null-terminated strings.
These specify the Instance and Class names to be used by both the client and the window manager for looking up resources for the application or as identifying information.
This property must be present when the window leaves the "Withdrawn" state and may be changed only while the window is in the "Withdrawn" state.
Window managers may examine the property only when they start up and when the window leaves the "Withdrawn" state, but there should be no need for a client to change its state dynamically.
For example, in table~\ref[tab:properties], we have seen a window with "WM_CLASS" set to {\em ``jetbrains-goland, jetbrains-goland''}.

%{\sbf WM\_CLIENT\_MACHINE} property should be set by the client to a string that forms the name of the machine running the client as seen from the machine running the server.
%It is not really useful for window managers.
%
%{\sbf WM\_COLORMAP\_WINDOWS} property on a top-level window is a list of the IDs of windows that may need colormaps installed that differ from the colormap of the top-level window.
%The window manager will watch this list of windows for changes in their colormap attributes.
%The top-level window is always (implicitly or explicitly) on the watch list.

\heading WM\uns HINTS
The "WM_HINTS" property is used to communicate information that does not need separate properties (such as "WM_CLASS")
to the window manager.
Its fields could be seen in table~\ref[tab:hints].
Window managers are free to assume convenient values for all fields of the "WM_HINTS" property if a window is mapped without one.
For our use case, particularly useful fields would be "flags", "initial_state", and "window_group".
Field "flags", apart from presence of all the other fields, contains boolean value for "UrgencyHint",
which is set to "true" when the window needs user attention.
The value of the "initial_state" field determines the state the client wishes to be in at the time the top-level window is mapped from the "Withdrawn" state,
that could be either "NormalState" (window is visible) or "IconicState" (icon is visible) -- see section~\ref[sec:toplevelwindow].
The "window_group" field lets the client specify that this window belongs to a group of windows.
Window manager can use this hint to manipulate the group as a whole.

\midinsert \clabel[tab:hints]{Contents of hints property}
\ctable{ll}{
    Field & Type \crl
    "flags" & "CARD32" \cr
    "input" & "CARD32" \cr
    "initial\_state" & "CARD32" \cr
    "icon\_pixmap" & "PIXMAP" \cr
    "icon\_window" & "WINDOW" \cr
    "icon\_x" & "INT32" \cr
    "icon\_y" & "INT32" \cr
    "icon\_mask" & "PIXMAP" \cr
    "window\_group" & "WINDOW" \cr
}
\caption/t Contents of {"WM_HINTS"}.
\endinsert

\heading WM\uns NAME
The "WM_NAME" property is an uninterpreted string that the client wants the window manager
to display in association with the window (for example, in a window headline bar).
Window managers are expected to make an effort to display this information, ignoring "WM_NAME" is not acceptable behavior.
Clients can assume that at least the first part of this string is visible to the user and that if the information is not visible to the user, it is because the user has taken an explicit action to make it invisible.
On the other hand, there is no guarantee that the user can see the "WM_NAME" string even if the window manager supports window headlines,
"WM_NAME" should not be therefore used for application-critical information or to announce asynchronous changes of an application's state that require timely user response.
The expected uses are to permit the user to identify one of a number of instances of the same client and to provide the user with noncritical state information.

\heading WM\uns ICON\uns NAME
The "WM_ICON_NAME" property is an uninterpreted string that the client wants to be displayed in association with the window when it is iconified (for example, in an icon label).
In other respects, including the type, it is similar to "WM_NAME".
For obvious geometric reasons, fewer characters will normally be visible in "WM_ICON_NAME" than "WM_NAME".

%\heading WM\uns ICON\uns SIZE
%property is set by by window manager that wishes to place constraints on the sizes of icon pixmaps and/or windows.
%The contents of this property are listed in the following table.
%
%\midinsert \clabel[tab:iconsize]{Contents of icon size property}
%\ctable{ll}{
%    Field & Type \crl
%    "min\_width" & "CARD32" \cr
%    "min\_height" & "CARD32" \cr
%    "max\_width" & "CARD32" \cr
%    "max\_height" & "CARD32" \cr
%    "width\_inc" & "CARD32" \cr
%    "height\_inc" & "CARD32" \cr
%}
%\caption/t Contents of {"ICON_SIZE"}.
%\endinsert

\heading WM\uns NORMAL\uns HINTS
The "WM_NORMAL_HINTS" property is of type "WM_SIZE_HINTS", its contents are displayed in table~\ref[tab:normalhints],
definitions of the "flags" field in table~\ref[tab:sizeflags].
To indicate that the size and position of the window (when a transition from the "Withdrawn" state occurs) was specified by the user,
the client should set the "USPosition" and "USSize" flags, which allow a window manager to know that the user specifically asked where
the window should be placed or how the window should be sized and that further interaction is superfluous.
To indicate that it was specified by the client without any user involvement, the client should set "PPosition" and "PSize".

The size specifiers refer to the width and height of the client's window excluding borders.
The "win_gravity" specifies how and whether the client window wants to be shifted to make room for the window manager frame.

The "min_width" and "min_height" elements specify the minimum size that the window can be for the client to be useful,
the "max_width" and "max_height" elements specify the maximum size.
The "base_width" and "base_height" elements in conjunction with "width_inc" and "height_inc" define an arithmetic
progression of preferred window widths and heights for non-negative integers $i$ and $j$:
$$width = base\_width + ( i \times width\_inc )$$
$$height = base\_height + ( j \times height\_inc )$$
Window managers are encouraged to use $i$ and $j$ instead of width and height in reporting window sizes to users.
If a base size is not provided, the minimum size is to be used in its place and vice versa.

The "min_aspect" and "max_aspect" fields are fractions with the numerator first and the denominator second, and they allow a client to specify the range of aspect ratios it prefers.
Window managers that honor aspect ratios should take into account the base size in determining the preferred window size.

\midinsert \clabel[tab:normalhints]{Contents of normal hints property}
\ctable{lll}{
    Field & Type & Fallback\crl \tskip4pt
    "flags" & "CARD32" & \cr
    "pad" & "4 * CARD32" & \cr
    "min\_width" & "INT32" & "base\_width" \cr
    "min\_height" & "INT32" & "base\_height" \cr
    "max\_width" & "INT32" & \cr
    "max\_height" & "INT32" & \cr
    "width\_inc" & "INT32" & \cr
    "height\_inc" & "INT32" & \cr
    "min\_aspect" & "(INT32,INT32)" & \cr
    "max\_aspect" & "(INT32,INT32)" & \cr
    "base\_width" & "INT32" & "min\_width" \cr
    "base\_height" & "INT32" & "min\_height" \cr
    "win\_gravity" & "INT32" & "NorthWest" \cr
}
\caption/t Contents of {"WM_NORMAL_HINTS"}.
\endinsert

\midinsert \clabel[tab:sizeflags]{Size hints flags definitions}
\ctable{lrl}{
    Name & Value & Field \crl \tskip4pt
    "USPosition" & 1 & User-specified x, y \cr
    "USSize" & 2 & User-specified width, height \cr
    "PPosition" & 4 & Program-specified position \cr
    "PSize" & 8 & Program-specified size \cr
    "PMinSize" & 16 & Program-specified minimum size \cr
    "PMaxSize" & 32 & Program-specified maximum size \cr
    "PResizeInc" & 64 & Program-specified resize increments \cr
    "PAspect" & 128 & Program-specified min and max aspect ratios \cr
    "PBaseSize" & 256 & Program-specified base size \cr
    "PWinGravity" & 512 & Program-specified window gravity \cr
}
\caption/t Size hints flags definitions.
\endinsert

\heading WM\uns PROTOCOLS
The "WM_PROTOCOLS" property is a list of atoms.
Each atom identifies a communication protocol between the client and the window manager in which the client is willing to participate.
Atoms can identify both standard protocols and private protocols specific to individual window managers.
The "WM_PROTOCOLS" property is not required.
If it is not present, the client does not want to participate in any window manager protocols.
There are three protocols defined by the ICCCM at the moment:
\begitems
* "WM_TAKE_FOCUS" -- assignment of input focus.
* "WM_DELETE_WINDOW" -- request to delete top level window.
* "WM_SAVE_YOURSELF" -- request to save client state (deprecated).
\enditems

\heading WM\uns STATE
The "WM_STATE" property is placed on each top-level client window that is not in the "Withdrawn" state by the window manager.
Top-level windows in the "Withdrawn" state may or may not have the "WM_STATE" property.
Once the top-level window has been withdrawn, the client may re-use it for another purpose.
Clients that do so should remove the "WM_STATE" property if it is still present.

The "WM_STATE" property is often used an an indicator of top-level window.
For example, some clients (such as xprop~\cite[xprop])
need to find a top-level window by pointer location (user clicking on window).
They can do it by searching the window hierarchy beneath the selected location for a window with the "WM_STATE" property.
This search must be recursive in order to cover all window manager reparenting possibilities.

The contents of the "WM_STATE" property are displayed in table~\ref[tab:wmstate]
\midinsert \clabel[tab:wmstate]{Contents of window state property}
\ctable{ll}{
    Field & Value \crl\tskip4pt
    "state" & one of {"WithdrawnState", "NormalState", "IconicState"}\cr
    "icon" & ID of icon window \cr
}
\caption/t Contents of {\tt WM\uns STATE} property.
\endinsert

\heading WM\uns TRANSIENT\uns FOR
The "WM_TRANSIENT_FOR" property might contain ID of another top-level window.
The implication is that this window is a pop-up on behalf of the named window, and window managers may decide
not to decorate transient windows or may treat them differently in other ways.
In particular, window managers should present newly mapped "WM_TRANSIENT_FOR" windows without requiring any user interaction,
even if mapping top-level windows normally does require interaction.
Dialogue boxes, for example, are an example of windows that should have "WM_TRANSIENT_FOR" set.

\label[sec:changingstate]
\secc Changing Window State
As already mentioned in section~\ref[sec:toplevelwindow], window manager will assign each top-level window one of three states:
\begitems
* "NormalState" -- top-level window is visible.
* "IconicState" -- top-level window is iconic.
That usually means that top-level window is not visible, but its "icon_window", "icon_pixmap" or "WM_ICON_NAME" is.
* "WithdrawnState" -- neither top-level window nor icon is visible.
\enditems
Newly created top-level windows are in the "Withdrawn" state.
Once the window has been provided with suitable properties, the client is free to change its state as follows:
\begitems
* "Withdrawn"\,\to\,"Normal" -- when window is mapped with "WM_HINTS.initial_state" being "NormalState".
* "Withdrawn"\,\to\,"Iconic" -- when window is mapped with "WM_HINTS.initial_state" being "IconicState".
* "Normal"\,\to\,"Iconic" -- clint sends "ClientMessage" event.
* "Normal"\,\to\,"Withdrawn" -- client unmaps window and sends a synthetic "UnmapNotify" event.
* "Iconic"\,\to\,"Normal" -- client maps the window, content of "WM_HINTS.initial_state" is irrelevant in this case.
* "Iconic"\,\to\,"Withdrawn" -- client unmaps window and sends a synthetic "UnmapNotify" event.
\enditems
Only the client can effect a transition into or out of the "Withdrawn" state.
Once a client's window has left the "Withdrawn" state, the window will be mapped if it is in the "Normal" state and the window will be unmapped if it is in the "Iconic" state.
Reparenting window managers must unmap the client's window when it is in the "Iconic" state, even if an ancestor window being unmapped renders the client's window unviewable.
Conversely, if a reparenting window manager renders the client's window unviewable by unmapping an ancestor, the client's window is by definition in the "Iconic" state and must also be unmapped.

\secc Configuring the Window
Clients can resize and reposition their top-level windows by using the "ConfigureWindow" request.
The attributes of the window that can be altered with this request are as follows:
\begitems
* The $[x,y]$ location of the window's upper left-outer corner.
* The width and height of the inner region of the window (excluding borders).
* The width of the border of the window.
* The window's position in the stack.
\enditems
The coordinate system in which the location is expressed is that of the root (irrespective of any reparenting that may have occurred).
The border width to be used and "win_gravity" position hint to be used are those most recently requested by the client.
Client configure requests are interpreted by the window manager in the same manner as the initial window geometry mapped from the "Withdrawn" state,
as described in the description of "WM_NORMAL_HINTS" property.
Clients must be aware that there is no guarantee that the window manager will allocate them the requested size or location
and must be prepared to deal with any size and location.
Window manager can respond to a "ConfigureRequest" request in three different ways:
\begitems
* Not change the size, location, border width, or stacking order of the window at all.\nl
A~client will receive a synthetic "ConfigureNotify" event that describes the (unchanged) geometry of the window.
The client will not receive a real "ConfigureNotify" event because no change has actually taken place.
* Move or restack the window without resizing it or changing its border width.\nl
A~client will receive a synthetic "ConfigureNotify" event following the change that describes the new geometry of the window.
The client may not receive a real "ConfigureNotify" event that describes this change because the window manager may have reparented the top-level window.
If the client does receive a real event, the synthetic event will follow the real one.
* Resize the window or change its border width.\nl
A~client will receive a real "ConfigureNotify" event, provided that it has selected for "StructureNotify" events.
The coordinates in this event are relative to the parent, which may not be the root if the window has been reparented.
\enditems
The general rule is that coordinates in real "ConfigureNotify" events are in the parent's space;
in synthetic events, they are in the root space.

Clients should be aware that their borders may not be visible.
Window managers are free to use reparenting techniques to decorate client's top-level windows with borders containing titles,
controls, and other details to maintain a consistent look-and-feel.
If they do, they are likely to override the client's attempts to set the border width and set it to zero.
Clients, therefore, should not depend on the top-level window's border being visible or use it to display any critical information.
Other window managers will allow the top-level windows border to be visible.
