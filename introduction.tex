\chap Introduction

Window manager, being a tool that controls the placement and appearance of windows on the screen,
is probably one of the most important parts of a graphical user interface of today's personal computers.
Generally, we can divide window managers into two categories, stacking window managers and tiling window managers.
Stacking window managers provide the traditional desktop metaphor and are the most widely used type.
They allow users to stack windows on top of each other, move them around, and resize them freely.
Tiling window managers, on the other hand, organize windows to some predefined, grid-like layouts.
Windows are sized so they cover the whole screen, do not overlap, and can't be moved or resized freely.
There are also window managers that combine both these approaches and are referred to as dynamic window managers.

While a lot of tiling window managers have been created in recent years, it is not the case for stacking window managers.
Available stacking window managers are either outdated, written in an unmaintainable and inextensible style, or they are
part of large desktop environments and with lots of dependencies.
Therefore, we feel a need for a small, easy to use, stacking window manager, that will be easily extensible.

%TODO some more ideas...
%While there are many tiling window managers
%While there is a plethora of tiling window managers available for the X Window System,
%the most widely used windowing system for Linux, situation is quite different for stacking window managers.
%Existing stacking window managers are either part of a big desktop environment
%There seems to be lack of a small, easy to use and extensible stacking window manager.
%Those existing are either large
%Stacking window managers are usually part of a big desktop environment, opinionated and not easily extensible.
%Those few existing stacking window managers that are small and meant to be used standalone,
%without the desktop environment, are quite old, implemented in C, mostly in a one big file,
%which makes them unmaintainable and inextensible.

\nonum
\sec Goals of the Thesis

The primary goal of this thesis is to design and implement a stacking window manager for the X11 Window System.
This window manager should be small, easy to use, and implemented in a modern system programming language.
It should be also compliant with two standard protocols defined on top of the X Window System Protocol,
Inter-Client Communication Conventions Manual (ICCCM), and Extended Window Manager Hints (EWMH).
Its core features should include:
\begitems
* stacking window management,
* basic window decorations,
* basic tiling (moving windows to half/thirds of the screen),
* extensibility,
* scriptability.
\enditems
The implementation should be tested.

\nonum
\sec Structure of the Thesis

The rest of the thesis is organised as follows.
In Chapter~\ref[chap:winman], we describe the windowing system, desktop environment, and window manager.
Different types of window managers are discussed and some window managers are described in detail.
In Chapter~\ref[chap:xsystem], we talk about the X Window System.
We discuss its history, architecture, and the X protocol.
In Chapter~\ref[chap:icccmewmh], two standard protocols defined on top of the X protocol, ICCCM and EWMH, are introduced.
Chapter~\ref[chap:impl] is the core part of the thesis, describing the design and implementation process.
Firstly, we go through the tools used to in the process.
Then, we describe the design choices, some implementation details, and finally how was the window manager tested.
