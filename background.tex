\label[chap:background]
\chap Background

\sec Windowing System

A~windowing system is a collection of software that creates the basic GUI (graphical user interface)
on computer display screens, including the drawing of windows and other graphics primitives for application programs~\cite[winsys].
From a programmer's point of view, a windowing system implements graphical primitives such as rendering fonts
or drawing a line on the screen, effectively providing an abstraction of the graphics hardware from higher-level
elements of the graphical interface like window managers~\cite[winsys2].

The most popular windowing system among Linux users is the {\em X Window System}~\cite[arch_xorg],
and its reference open-source implementation {\em X.Org Server} provided by the X.Org Foundation~\cite[xorg].
The X Window System (also referred to as X or X11) originated at MIT in 1984~\cite[xman2].
Another windowing system is {\em Wayland}, which is intended as a simpler replacement for X~\cite[wayland].
Wayland started in 2008 and comes with different architecture, trying to solve problems in X's approach.

The main problem in X's approach that Wayland aims to solve, is that it does not isolate applications from each other.
As a result, all X applications have access to everything on the screen, all can register to receive every keystroke,
even if they do not have input focus, and they can even inject keystrokes into other windows~\cite[xproblems].
The architecture schemes of X and Wayland are depicted in Figures~\ref[fig:xarch] and~\ref[fig:wayarch].
The numbers show the flow of the events from the input device to the point where the change it affects appears on the screen.
The only difference between those two schemes is the compositor,
which is responsible for rendering the entire screen contents.
Since the window location on the screen is controlled by the compositor and may be transformed
in several ways (e.g., scaled-down, rotated), the X server does not have the information to decide which client
should receive the event.
The X server acts as a middleman that introduces an extra step between applications and the compositor
and an extra step between the compositor and the hardware.
In Wayland, on the other hand, the compositor is part of the display server.
The Wayland protocol lets the compositor send the input events directly to the clients and lets
the client send the response directly to the compositor~\cite[wayarch].

\midinsert
\line{\hsize=.5\hsize \vtop{%
    \clabel[fig:xarch]{X architecture scheme}
    \picw=7.7cm \cinspic img/xarch.pdf
    \caption/f X architecture scheme
    \par}\vtop{%
    \clabel[fig:wayarch]{Wayland architecture scheme}
    \picw=6.6cm \cinspic img/wayarch.pdf
    \caption/f Wayland architecture scheme
    \par}}
\endinsert

Even though Wayland is meant to replace X in the future, it is not happening quite yet.
One of its problems is apps compatibility -- software engineer Samuel Walladge said in his 2019 article
``Are we Wayland yet?''~\cite[arewewayland]: ``I soon discovered that many many apps either only supported X11, or crashed on Wayland.''
Those apps then require XWayland, which provides an embedded X server for apps that do not support Wayland yet.
Another problem is that ``screen recording or sharing apps just do not work''~\cite[arewewayland],
which is caused by the Wayland architecture that isolates each client.
Walladge also mentions stability issues and concludes that the time for switching to Wayland did not come yet.
Similar problems and conclusions are also outlined in articles ``X11 Sucks… So What's Up With Wayland?''~\cite[x11sucks]
and ``Wayland v/s Xorg : How Are They Similar \& How Are They Different''~\cite[wayvsxorg].
Because of that, we decided to build swm for the X Window System.

\sec Desktop Environment and Window Manager

A~desktop environment is an implementation of the desktop metaphor,
which was introduced by Alan Kay at Xerox PARC in 1970 to help users interact with the computer more easily~\cite[graphics].
The first commercial personal computer to use the desktop metaphor was the Xerox Star~\cite[alto].
The developer of its interface, David Smith, described it in 1982 article~\cite[smith]:
``Every user’s initial view of the Star is the desktop, which resembles the top of an office desk,
together with surrounding furniture and equipment.
It represents a working environment, where projects and accessible resources reside.
On the screen are displayed pictures of familiar office objects, such as documents, folders, file drawers, in-baskets, and out-baskets.
These objects are displayed as small pictures, or icons.''

A~desktop environment, as we know it today, bundles together a variety of components to provide common
graphical user interface elements such as icons, panels, wallpapers, and desktop widgets~\cite[arch_de].
Most desktop environments also include a set of integrated applications and utilities, such as text editor, file manager, or web browser.
One of the most important parts of a desktop environment is a window manager.
While the desktop environment provides its own window manager, it can be usually replaced with another compatible one~\cite[arch_de].
There is a great offer of desktop environments for Linux, for example, Arch Wiki~\cite[arch_de] lists more than 20 of them.
Some of the most popular are KDE~\cite[kde], GNOME~\cite[gnome], or Xfce~\cite[xfce].

A~window manager is system software that manages windows.
That means, it controls the placement and appearance of windows within a windowing system in GUI~\cite[arch_wm].
It allows windows to be opened, closed, resized, moved, maximized, minimized, and more.
The window manager is also responsible for tracking which window is currently active and thus receiving the user's input.
Some window managers are specifically developed to be part of a desktop environment, others are instead designed to be used standalone.
Using a standalone window manager allows the user to create a more lightweight and customized environment, tailored to his own specific needs~\cite[arch_wm].
There is a great offer of window managers for Linux.
For example, Arch Wiki provides a list of window managers categorized by type~\cite[arch_wm], that lists more than 60 of them.
In Section~\ref[sec:managers], we will pick some of them and discuss them in more detail.

%TODO: describe the difference between DE and lightweight desktops with just wm

\sec Types of Window Managers

Window managers are usually divided into two classes, based on how windows are drawn on the screen.

\secc Stacking Window Managers

Stacking window managers, also known as floating, provide the traditional desktop metaphor used in commercial operating systems like Windows and OS X~\cite[arch_wm].
Windows act like pieces of paper on a desk -- they can be stacked on top of each other, moved around, and resized freely by the user.
Moving or resizing one window does not affect the position or size of other windows, only their visible area.
Window manager must maintain the stacking order of windows and only the top window on the stack is guaranteed to be fully visible.
A window is usually moved to the top of the stack by the window manager when the user starts to interact with it.
Example layout of a stacking window manager is depicted in Figure~\ref[fig:stacking].
Windows are overlapping and only focused window (highlighted in blue) is fully visible.

\midinsert
\picw=8cm \cinspic img/stacking.pdf
\clabel[fig:stacking]{Stacking window manager layout}
\caption/f The layout of a stacking window manager
\endinsert

\secc Tiling Window Managers

Tiling window managers tile the windows so that none are overlapping and they usually use all the available screen space.
They typically make very extensive use of keyboard shortcuts and have less or no reliance on the mouse~\cite[arch_wm].
It is not possible to move or resize windows freely -- enlarging a window shrinks its adjacent windows and vice versa,
moving is typically done by swapping the window's position with another window.
Some types of windows, such as dialogs and pop-up windows, are not suited for tiling though.
Most tiling window managers detect those windows and leave them in a floating state, keeping them above
the windows that are tiled.
Example layout of a tiling window manager is depicted in Figure~\ref[fig:tiling].

Tiling window managers could be split into two more categories:
\begitems
* {\sbf Dynamic} tiling window managers tile windows based on preset layouts.
They usually offer many different layouts of window placement and the user can dynamically switch between them.
For example, one of the most common tiling layouts is called {\em master-stack} layout,
in which one window (master) is considered the most important and has dedicated half of the screen space,
while the rest of the windows are displayed one below the other on the second half of the screen.
* {\sbf Static (manual)} tiling window managers do not use layouts.
They let the user decide where windows should be placed.
Most typically, when a new window is created, one of the existing windows is shrunk to half its width/height and the new window takes freed up space.
The user can choose the window that will be shrunk and also the direction (vertical or horizontal).
Sometimes, the user can also choose the default split ratio, so that the new window does not take half, but, for example,
only a third of the existing window space.
\enditems

\midinsert
\picw=8cm \cinspic img/tiling.pdf
\clabel[fig:tiling]{Tiling window manager layout}
\caption/f The layout of a tiling window manager
\endinsert

\label[sec:managers]
\sec Window Managers

As mentioned before, there is a great offer of window managers for Linux.
In this section, we will go through some of them, mainly those that inspired swm in some way.

\secc Cwm

Cwm (Calm Window Manager) is a stacking window manager for the X Window System.
It describes itself as a ``window manager which contains many features that concentrate on the efficiency
and transparency of window management while maintaining the simplest and most pleasant aesthetic''~\cite[cwm].
It is the default window manager on OpenBSD.

Cwm is oriented towards heavy keyboard usage -- resizing, moving, hiding, raising, or lowering windows, all can be done using
keyboard shortcuts~\cite[cwm2].
It could be even configured to move the mouse cursor using the keyboard so that one could use the computer without any pointing device.
The user can either define custom shortcuts using cwm's configuration file or use the defaults.

Interface of cwm is very minimal -- it only draws a one-pixel border around windows by default,
the width of the border and its color can be configured though.
Cwm offers several menus, which can be used to launch applications or switch between running applications.
The principle of these menus is that cwm shows a list of relevant content and user can search in it and finally pick one of the filtered options~\cite[cwm3].
Windows are searched by their current title, old titles and by their label, which is a custom string user can assign to every window.

Instead of the traditional virtual desktop concept, cwm is using groups.
There are nine groups with IDs 1--9 and a special ``sticky'' group with ID 0.
Each window is assigned to one of those groups, the user can then change its group with a keyboard shortcut.
While the sticky group is always visible on the screen, the normal group can be either visible or hidden.
This means that unlike virtual desktops, multiple groups could be visible at the same time.
The visibility of the group can be controlled by keyboard shortcuts as well.

Figure~\ref[fig:cwmmenu] shows a cwm desktop with two windows and one of those menus.
This menu lists all the managed top-level windows, there are various information included:
\begitems
* the group ID on the left,
* ``!'' indicates active (focused) window,
* ``\&'' indicates hidden windows,
* the label of the window in square brackets,
* and finally the title of the window.
\enditems

All the cwm's configuration is done using configuration file ".cwmrc".
The user may add, modify or remove shortcuts, change colors and other aspects of window management behavior~\cite[cwm3].
One of the drawbacks of cwm is that it does not support most of the EWMH protocol.

\midinsert
\picw=12cm \cinspic img/cwm.png
\clabel[fig:cwmmenu]{Cwm desktop}
\caption/f Cwm desktop with two windows and a menu showing all managed windows
\endinsert

\secc Openbox

Openbox is a stacking window manager.
It describes itself as a ``minimalistic, highly configurable, next generation window manager with extensive standards support''~\cite[openbox].
It is designed to be fully compliant with ICCCM and EWMH protocols.
It is the default window manager of desktop environments LXDE~\cite[lxde] and its successor LXQt~\cite[lxqt],
and it is also offered as one of the options for default window manager in many Linux distributions,
for example Lubuntu~\cite[lubuntu] and Manjaro~\cite[manjaro].

Although Openbox claims to be minimalistic, it is much more advanced than cwm.
Its default UI could be seen in Figure~\ref[fig:openbox].
It comes with full-featured window decorations that include a border around the window and title bar with window title
and buttons to minimize, maximize, and close the window.
It also includes a {\em root menu} on the desktop.
Using this menu, one can launch applications, bring minimized windows up again, or start various Openbox configuration tools.
Openbox does not come with a bundled taskbar, though, and advices its users to use one of many stand-alone EWMH compatible taskbars.

Openbox is configured by two configuration files, "rc.xml" and "menu.xml".
The first one ("rc.xml") is the main configuration file, responsible for determining the behavior and settings
of the overall session, including keyboard shortcuts, theming, virtual desktops settings etc.~\cite[openboxarch]
The second one ("menu.xml") defines the type, behavior, and items of aforementioned desktop menu.
Although the default provided is a static menu (it will not automatically update when new applications are installed),
it is possible to employ the use of dynamic menus that will automatically update as well~\cite[openboxarch].

Configuration files are not meant to be easily readable or editable -- Openbox comes with applications to edit them,
one for "rc.xml", called {\em obconf} and one for "menu.xml", called {\em obmenu}.
Using obconf, user can, for example, easily switch Openbox theme, that controls the looks of window decorations and menus.
Obconf itself includes many themes to choose from, more themes are available online and one can also
easily create new or modify existing theme using provided GUI application~\cite[openboxarch].

\midinsert
\picw=12cm \cinspic img/openbox.png
\clabel[fig:openbox]{Openbox windows and menu}
\caption/f Openbox with two opened window and a menu
\endinsert

\secc Dwm

Dwm (Dynamic Window Manager) is a dynamic tiling window manager.
It manages windows in tiled, monocle and floating layouts.
All of the layouts can be applied dynamically, optimising the environment for the application in use and the task performed.~\cite[dwm]

Dwm follows the philosophy of its authors, {\em suckless.org}, that is about keeping things simple, minimal and usable~\cite[suckless].
All their projects focus on advanced and experienced computer users.
They believe that ingenious software is simple and that as the number of lines of code in software shrinks,
the less the software sucks~\cite[suckless].
Following this philosophy, dwm's source code is intended to never exceed 2\,000 source lines of code
and it has no configuration file, it can only be configured by editing its C source code~\cite[dwm].

Dwm is extremely lightweight and fast and its interface is very minimal.
It only draws a small customizable border around windows to indicate the focus state.
Windows could be spread across nine desktops called tags and it is possible to show windows from multiple tags at once.
It comes with its own status bar which displays all available tags,
the layout, the number of visible windows, the title of the focused window, and the text read from the root window name property~\cite[dwm].
It is not really possible to use third party taskbars, because dwm does not support EWMH.

As stated before, dwm comes with three layouts by default.
In tiled layout, windows are managed in a master and stacking area.
The master area contains the window which currently needs most attention, whereas the stacking area contains all other windows.
In monocle layout, all windows are maximized to the screen size and in floating layout, windows can be resized and moved freely.

Interesting concept of dwm are {\em patches}.
Since the core dwm source code is very minimal and is kept under 2\,000 lines of code,
a lot of functionality is missing, and so users started to create patches to add functionality they wanted.
Those patches are in a form of simple git diff files\urlnote{https://git-scm.com/docs/git-diff},
and could be found on dwm's project website~\cite[dwmpatch].
Most of those plugins implement new types of layout (e.g., deck, fibonacci, grid), or modifies behavior or style of the status bar.

\midinsert
\picw=12cm \cinspic img/dwm.png
\clabel[fig:dwmdesktop]{Dwm desktop}
\caption/f Dwm desktop with the status bar and four open windows in its default layout
\endinsert

\secc Bspwm

Bspwm (binary space partitioning window manager) is a manual tiling window manager.
It represents windows as the leaves of a full binary tree, supports multiple monitors and is configured and controlled through messages~\cite[bspwm, bspwmarch].
It supports a subset of ICCCM and EWMH protocols, it does not state what is supported and what is not, though~\cite[bspwm].

Bspwm is very minimalistic in terms of UI -- it only draws a simple solid color border around its windows and has no build-in taskbar or menus.
Bspwm only responds to X events, and messages it receives on a dedicated socket.
It comes with a standalone command-line application called {\em bspc}, that writes messages on the bspwm's socket~\cite[bspwm].
It also does not handle any keyboard or pointer inputs.
For this it relies on a third party program that translates keyboard and pointer events to bspc invocations~\cite[bspwm].
It recommends sxhkd~\cite[sxhkd] (developed by the same author).
Upon startup, bspwm runs its configuration file, {\em bspwmrc}, which is simply a shell script that calls bspc (and, possibly, other utilities)~\cite[bspwm].

Since bspwm represents windows as the leaves of a full binary tree,
each inner node has exactly two children and each leaf node holds exactly one window.
Each inner node is responsible for splitting its screen space in two parts and this split is defined by two
parameters: the type (horizontal or vertical) and the ratio (a real number from interval $(0,1)$)~\cite[bspwm].
New windows are inserted into a window tree at the specified insertion point using specified insertion mode, which
is configurable~\cite[bspwm].

\midinsert
\picw=12cm \cinspic img/bspwm.png
\clabel[fig:bspwmdesktop]{Bspwm desktop}
\caption/f Bspwm desktop with four open windows in its default layout
\endinsert

\secc i3

i3 is a dynamic tiling window manager that is primarily targeted at developers and advanced users~\cite[i3].
It uses a tree as a data structure for windows, which, according to its authors, allows for more flexible layouts
than the column-based approach used by other window managers~\cite[i3].

Unlike dwm, i3 does not try to be as minimal as possible, but it does not want to be bloated either.
This is described in one of its initial goals~\cite[i3] like this: ``Don’t be bloated, don’t be fancy
(simple borders are the most decoration we want to have).
However, we do not enforce unnecessary limits such as a maximum amount of source lines of code.
If it needs to be a bit bigger, it will be.''

The default UI of i3wm is depicted in Figure~\ref[fig:i3desktop].
It is not that simple as UI of dwm or bspwm, but it is not very complex either.
Windows have a simple border and also a title bar, that only shows the name of the window.
It also comes with its panel called {\em i3bar}, which is used by default (see screenshot).
Since i3 supports EWMH, any other standalone panel could be used as well, though.

i3 offers three layouts for windows:
\begitems
* split vertically/horizontally -- windows are sized so that every window gets an equal amount of space.
This can be done either vertically or horizontally.
* stacking -- only focused window is displayed.
On top of it is a list of the rest of the windows.
* tabbed -- only focused window is displayed.
On top of it are tabs representing the rest of the windows.
\enditems
All those layouts can be combined on the screen using so-called containers.
This means that inside a container that is tabbed can be another container that is using a split layout, for example.
An example is depicted in Figure~\ref[fig:i3desktop].
The leftmost window is in fact a container that is using a tabbed layout and has two windows.
Similarly, the container right next to it has two windows in a stacking layout.
A window can be also completely removed from the tiling layout and managed as a floating window.

i3 is highly configurable using its configuration file.
User can configure the UI (border width and color for multiple window states, fonts, etc.),
keyboard shortcuts, i3bar, and the behavior of the window manager overall.
For example, the user can choose that windows will not be focused upon opening, or that
specific window will be in a floating state by default.

\midinsert
\picw=12cm \cinspic img/i3wm.png
\clabel[fig:i3desktop]{i3 desktop}
\caption/f Desktop of i3 with four containers and six open windows
\endinsert

\secc Summary

In this section, we discussed five window managers.
We picked some from each category -- cwm and Openbox represented stacking window managers,
dwm, bspwm, and i3 represented tiling window managers.
In each category, we also showed different approaches.
Cwm and dwm represent minimal window managers with very minimal UI, configuration, and no EWMH support.
Bspwm stands somewhere in the middle, with its minimal UI, but quite advanced functionality and EWMH support.
Finally, Openbox and i3 represent window managers with advanced UI (full-fledged themeable window decorations in Openbox,
title bars and tabs in i3) and advanced configuration.

There are many more window managers out there, both stacking and tiling, their functionalities are not much different from
those presented here, though.
To mention a few more window managers, we can name Xfwm~\cite[xfce] or Kwin~\cite[kde], which are both stacking window managers
and both are developed as a part of a desktop environment, Xfce and KDE, respectively.
For tiling window managers, we can mention awesome~\cite[awesome] or xmonad~\cite[xmonad], which were both inspired by dwm and they build upon it.
Another popular tiling window managers are herbstluftwm~\cite[herbst] or Qtile~\cite[qtile].
For an extensive list of both stacking and tiling window managers, we can recommend Arch Linux Wiki~\cite[arch_wm].

\input xsystem
\input icccm
\input ewmh
