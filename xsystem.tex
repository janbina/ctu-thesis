\label[chap:xsystem]
\label[sec:xsystem]
\sec The X Window System
The X Window System is a network transparent window system which runs on a wide range of computing and graphics machines.~\cite[xman]

In this section, we will cover how the X server works, with a focus on communication with its clients.
We will not cover how the X server handles resources, color, graphics, text or input as it is not
that important for window management.

\secc History
The X Window System was created in the mid-1980s at the Massachusetts Institute of Technology.
In 1988, MIT formed a member-funded consortium to provide the technical and administrative leadership necessary
to support further development of the X Window System.
In 1992, MIT and the membership decided it was in their best interests to move the consortium out of MIT and create an independent, stand-alone organization.
All rights to the X Window System were assigned by MIT to X Consortium, Inc. on January 1, 1994.
On December 31, 1996 the X Consortium, Inc. closed its doors and all rights to the X Window System were assigned to The Open Group (then known as the Open Software Foundation).
The X.Org Foundation was formed in 2004 as the successor to the X.Org Group at The Open Group.
The purpose of the X.Org Foundation is to foster the development, evolution, and maintenance of the X Window System,
a comprehensive set of vendor-neutral, system-architecture neutral, network-transparent windowing and user interface standards.
Membership in the X.Org Foundation is free and open to anyone.
The X.Org Foundation hosts a public git repository of the source code on \url{freedesktop.org}.~\cite[xman2]

The X protocol has been at version 11 since September 1987.
Many revisions has been released since then, the latest one being X11R7.7 released in June 2012.~\cite[xreleases]

Version 11 is a complete window programming package.
It offers much more flexibility in the areas of supported display
features, window manager styles, and support for multiple screens and provides better performance than X Version 10.~\cite[xguide1]

\label[sec:xarch]
\secc Architecture
X uses a client-server model.
The server is the software that manages one display, keyboard, and mouse.
The client is a program displaying on the screen and taking input from that keyboard and mouse.
A~client sends drawing drawing requests and information requests to the server, and the server sends back to the client user input,
replies to information requests, and error reports.
The client may be running on the same machine as the server or on a different machine over the network.~\cite[xguide0]

The X Window System is not limited to a single client interacting with a single server.
The most common scenario is that many clients are interacting with a single server, which is the case
when several applications are displaying on a single display.
Also, a single client can communicate with several servers.~\cite[xguide0]

The server acts as an intermediary between user programs (called clients or applications) running on either the local or
remote systems and the resources of the local system.
The server performs the following tasks:~\cite[xguide1]
\begitems
* Allows access to the display by multiple clients.
* Interprets network messages from clients.
* Passes user input to the clients by sending network messages.
* Does two−dimensional drawing -- graphics are performed by the display server rather than by the client.
* Maintains complex data structures, including windows, cursors, fonts as resources that can
be shared between clients and referred to simply by resource IDs. Server−maintained resources reduce the amount of
data that has to be maintained by each client and the amount of data that has to be transferred over the network.
\enditems

The window manager is a client that has authority over the layout of windows on the screen.
Certain X protocol features are used only by the window manager to enforce this authority.
Otherwise, the window manager is just like any other client.~\cite[xguide0]

\secc The X Protocol
The X protocol is the true definition of the X Window System, and any code in any language that implements it is a true implementation of X.
It is designed to communicate all the information necessary to operate a window system over a single asynchronous bidirectional stream of bytes.~\cite[xguide0]

Below the X protocol, any lower layer of network can be used, as long as it is bidirectional
and delivers bytes in sequence and unduplicated between a server process and a client process.
When the client and server are on the same machine, the connection is based on local interprocess communication (IPC) channels,
shared memory, or UNIX domain sockets.
However, even local connections use the X protocol.~\cite[xguide0]

\secc Message Types

The X protocol specifies four types of messages -- requests, replies, events and errors.
While requests are sent from the client to the server, replies, events and errors are sent from the server to the client.
\begitems
* {\sbf Request} is generated by the client and sent to the server.
It can carry wide variety of information, such as an inquiry about the current size of the window
* {\sbf Reply} is sent from the server to the client in response to certain requests.
Not all requests are answered by replies though -- only the ones that ask for information.
* {\sbf Event} is sent from the server to the client.
Contains information about a device action or about a side effect of a previous request.
* {\sbf Error} is like an event, but it is handled differently by clients.
\enditems
There are two types of requests, depending whether it requires reply or not.
Request that requires a reply is called a round-trip request.
Round-trip requests should be minimized in client programs, because they lower performance in case of network delays.~\cite[xguide0]

\label[sec:propertiesatoms]
\secc Properties and Atoms
The developers of X needed a way to allow clients to communicate arbitrary data with each other, and they came up with properties.
A~property is a packet of information associated with a window, made available to all the clients running under a server.
Properties are used by clients to store information that other clients might need or want to know and to read that information when set by other clients.~\cite[xguide1]

Each property has a unique integer ID, called an atom, and name, which is an ASCII string.
An atom is just a nickname for a property, so that arbitrary length property name strings
do not have to be transferred back and forth between the client and the server.
The atom is assigned by the server and will remain defined in the server even after the client that defined it terminates.
A~property is uniquely identified by an atom and a window.
Each property has a type, which itself is a property.
There are several predefined properties for use as some of the more often needed types,
such as "STRING", "WINDOW", or "CARDINAL".
~\cite[xguide1]

The data associated with a property can be stored as an array of 8−bit quantities, 16−bit quantities, or 32−bit quantities only.
Properties can contain structures or raw data, but if one is to contain a structure of complex type, it must be encoded
into one of the three byte formats by the program before being sent to the server and decoded when read from the server.~\cite[xguide1]

Properties might be set even on the root window.
Those properties remain defined even after the client that set them has exited.~\cite[xguide1]

One of the most important uses of properties is to communicate information from applications to the window manager and vice versa.
The application sets properties on its top−level window to specify for example the range of sizes it prefers for
its top−level window and other information.
Properties also communicate the other way; for example, the window manager specifies what sizes of icon pixmaps it prefers.~\cite[xguide1]

The table~\ref[tab:properties] shows some of the properties retrieved from
the top-level window of an application using the {\em xprop} utility~\cite[xprop].
While properties like "WM_CLASS" and "WM_NAME" were set by the application itself,
property "_NET_WM_ALLOWED_ACTIONS" is set exclusively by the window manager and
property "_NET_WM_DESKTOP" could be set by both of them.
We will discuss the meaning of some significant properties later on.

\midinsert \clabel[tab:properties]{Window properties}
\ctable{ll}{
    Property name and type & Value\crl\tskip4pt
    "WM\_STATE(WM\_STATE)" & window state: Normal \cr
                           & icon window: 0x0 \cr
    "\_NET\_WM\_ALLOWED\_ACTIONS(ATOM)" & "\_NET\_WM\_ACTION\_MOVE, ..." \cr
    "\_NET\_WM\_DESKTOP(CARDINAL)" & 1 \cr
    "\_NET\_WM\_WINDOW\_TYPE(ATOM)" & "\_NET\_WM\_WINDOW\_TYPE\_NORMAL" \cr
    "\_NET\_WM\_STATE(ATOM)" & "\_NET\_WM\_STATE\_MAXIMIZED\_VERT, ..." \cr
    "WM\_PROTOCOLS(ATOM)" & "WM\_DELETE\_WINDOW, WM\_TAKE\_FOCUS" \cr
    "WM\_CLASS(STRING)" & jetbrains-goland, jetbrains-goland \cr
    "\_NET\_WM\_NAME(UTF8\_STRING)" & swm [\char`\~/projects/swm] - .../cmd/swm/main.go \cr
    "WM\_NAME(STRING)" & swm [\char`\~/projects/swm] - .../cmd/swm/main.go \cr
}
\caption/t Window properties retrieved using xprop~\cite[xprop].
\endinsert


\secc Events
An event is a packet of information that is generated by the server when certain actions occur.
For example, moving the pointer or pressing the key is an event.
Here are some other examples of events:~\cite[xguide1]
\begitems
* Mouse button pressed or released -- "ButtonPress", "ButtonRelease".
* Window mapped or unmapped -- "MapNotify", "UnmapNotify".
* Mouse crossing a window boundary -- "EnterNotify", "LeaveNotify".
\enditems
A~client must select the event types that it wants the server to send for each window.
For example, a scrollbar may require mouse button events but not keyboard events, while the main window of an
application may require keyboard but not mouse events.
One would select different event types on each of these windows.~\cite[xguide1]

Every event contains the "type" field, which specifies the type of the event, "window" field, which
is the ID of the window to which the event was sent, and other information specific to the event type.
Because events can arrive in any order, the structure of code to handle them is predetermined.
Every program contains an event loop in which each event is received and processed.
Normally this loop is implemented as an infinite while loop,
beginning with an event−getting routine and followed by a switch statement that branches according to the event type.
Within each branch for an event type, there may be additional branches corresponding to the window in which the event
occurred or other fields in the event structure.~\cite[xguide1]

\label[sec:xwinhierarchy]
\secc Window Hierarchy

X windows are arranged in a tree hierarchy.
Each window always has exactly one parent window, which is assigned as the window is created.
The only window that has no parent is called the {\em root window} and fills the entire screen.
The root window is created by the X server as it starts up.~\cite[xguide1]

The first windows to be created by each client are children of the root window.
The children of the root window are special -- they are called top-level windows and they are managed by the window manager.
Each child may also have its own child windows.
These child windows of the top−level windows are used to create application features like buttons and text boxes.
Each window is contained within the limits of its parent and each window has its own coordinate system.
The origin of a window is the top−left corner of the window and the x and y coordinates increase to the right and bottom.~\cite[xguide1]

\secc The Window Manager

The window manager is just another client of X server, but by convention, it is given special responsibilities.
Much of the communication between clients and the window manager and vice versa occurs through properties, the rest is
occurring through events.
Many of the properties are known as hints because they may not necessarily be honored by the
window manager, even if one is running.
An application must be prepared for the window manager to ignore, modify, or
honor the preferences it indicates through the window manager hints.~\cite[xguide1]

\secc X Client Libraries

Two official helper libraries that provide API for talking to the X Server exist, xlib and XCB.

\heading Xlib
Xlib, also known as libX11, is the original C language X11 API, released in 1985.
It was designed to look like a traditional library API, hiding the fact that calls result
in protocol requests to a server.
Calls that don't require a response from the X server are queued in a buffer
to be sent as a batch of requests to the server.
Those that require a response flush all the buffered requests and then
block until the response is received.
This mix of synchronous and asynchronous behavior causes some problems,
because it is not obvious which calls implicitly flush the buffer and which do not.\cite[XlibXCB]

\heading XCB
XCB is a second attempt at defining a C language binding for X11.
It was first released in 2001, after many years of experience with Xlib,
learning from it and other protocol interface libraries.
XCB makes the client-server nature of the protocol explicit in its design.
The client is in charge of deciding when to flush the request buffer,
when to read results and when to wait for the server to respond.\cite[XlibXCB]

\heading Comparison
In code listings~\ref[code:xlib] and~\ref[code:xcb], we can see comparison of those two libraries on the task of looking up a window property.
Xlib generates the request to the X server to retrieve the property and appends it to its buffer of requests.
Since this is a request that requires a response, Xlib then flushes the buffer, sending the contents to the X server.
Next, Xlib waits until the X server processes all the requests preceding the property retrieve request, and sends the property retrieve reply.
Xlib then returns the reply to the client.
XCB, on the other hand, provides functions generated directly from the protocol descriptions in an obvious mechanistic way.
XCB functions map directly onto the protocol, with separate functions to put requests into the outgoing buffer and to read results back from the X server asynchronously later.
The power of XCB is in allowing those two steps to have as much code as you want between them.
The programmer decides when to wait for data, instead of being forced to wait for the data returned by a request at the time the request is issued.\cite[XlibXCB]

\codeblock{code:xlib}{Window property lookup using Xlib}{winproperty_xlib.c}{GO}
\codeblock{code:xcb}{Window property lookup using XCB}{winproperty_xcb.c}{GO}

Xlib and XCB are compatible, meaning that one can mix calls to the first with calls to the other.
This compatibility was achieved by rebuilding libX11 as a layer on top of libxcb.
Xlib and XCB share the same X server connection and pass control of it back and forth.
That option was introduced in libX11 version 1.2, and since version 1.4, released in 2010, it is always present.\cite[XlibXCB]

Most applications should call Xlib and XCB sparingly.
Higher level toolkits provide more efficient programming models, and support features expected in modern applications,
including support for complex internationalized input and output, accessibility, and integration with desktop environments.
However, sometimes applications will find themselves needing to make calls to the raw underlying X11 libraries for operations not supported by toolkits.
For example, an application might need to make calls to X extension API's not covered in the current version of the toolkit's programming model.\cite[XlibXCB]

For the implementation of swm, we used the X Go Binding library~\cite[xgb], which is based on XCB.
We will talk about it in more detail in chapter~\ref[chap:impl].
