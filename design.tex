\label[chap:design]
\chap Design

% ######################################################################################################################
%\sec Design % ##########################################################################################################
% ######################################################################################################################

The primary design goal for SWM was to be {\em ``small, easy to use yet hackable stacking window manager.''}
To design such window manager, we based it on one of the core principles of Unix philosophy, that programs should do one thing well,
as stated by Doug McIlroy, one of the founders of the Unix tradition, in the Bell System Technical Journal~\cite[mcilroy]:
{\em ``Make each program do one thing well.
To do a new job, build afresh rather than complicate old programs by adding new features.''}

To honor this principle, SWM does not come with its own taskbar or application launcher, as some window managers do.
Instead, it focuses on that one thing, window management, and the rest is left for third party applications.
Thanks to ICCCM and EWMH specifications, third party applications can interact with SWM seamlessly
and user can choose any application that fits his needs instead of using the one bundled with window manager.

For example, there are many open source taskbar implementations,
like polybar~\cite[polybar], lemonbar~\cite[lemonbar], or pypanel~\cite[pypanel], just to name a few.
To mention some application launchers that could be used with SWM, we can name dmenu~\cite[dmenu] or rofi~\cite[rofi].

In this section, we will discuss some of the design choices made in development of swm into detail.

% ######################################################################################################################
\sec Configuration and Controlling % ##################################################################################
% ######################################################################################################################

Most of the window managers are using configuration files with lots and lots of options you can configure.
With swm, we decided to take a different approach:
\begitems
* There is no need for swm to have commands for actions like maximizing or minimizing the window -- since it supports EWMH
and there are tools for sending EWMH messages (xdotool, wmctrl), we can just make use of them.
* There is no need to handle keyboard shortcuts -- users can use sxhkd that does it better than we could ever do.
\enditems
This means that to maximize and minimize the window, as well as to do many more actions, user can use sxhkd in combination
with xdotool or wmctrl and swm does not have to be involved at all.

Not everything could be done through EWMH and third party tools, though.
To address those cases, swm comes with two tools -- swmctl and swmrc -- that we will discuss in the next section.

% ######################################################################################################################
\sec Swmctl and Swmrc % ###############################################################################################
% ######################################################################################################################

Swmctl is a simple command line utility that can be used to send commands to swm.
It is primarily meant to be used in combination with sxhkd to invoke the commands with keyboard shortcuts,
but it can also be used directly from command line.
This makes it easy to test the command first and edit the configuration file only after it works exactly how one wants it to.

All the swmctl's commands will be discussed one by one in the {\em implementation}~\ref[sec:impl] section later,
but to get an idea of how swmctl works and how it is better than simple configuration file, lets have a look at
following example.

To move a window using keyboard in cwm, one has to bind keyboard shortcuts in its configuration file like this:
\begtt
moveamount 1
bind CM-k  window-move-up
bind CMS-k window-move-up-big
\endtt
As you can see, there is an option "moveamount", which defines how many pixels the window will move.
Then, there are two commands for each direction "window-move-X" and "window-move-X-big".
The first one moves the window by "moveamount" pixels, the second one moves it by ten times "moveamount" pixels.

In swmctl, on the other hand, there is a "move" command, which, by an argument takes the direction(s) and amount of pixels to move the window by.
User can then bind this command to keyboard shortcut using sxhkd like this:
\begtt
ctrl + alt + {h,j,k,l}
swmctl move -{w,s,n,e} 20
\endtt

This approach is both much more simple and versatile than cwm's:
\begitems
* while cwm is limited to two move amounts (and they are even tight together), swm can handle arbitrary amount,
* swm can move the window in multiple directions in one command, and it can even be moved by different amount of pixels in each direction.
\enditems

Swmrc is just a script that is executed upon startup of swm.
It is executed after all the initialization has been done, so the window manager is running and
can accept any command, be it swmctl's one or EWMH message sent using xdotool, but right before all the existing windows
are adopted by swm.
This makes it ideal for all configuration commands -- one can, for example, set the names of the groups, color of borders etc.

% ######################################################################################################################
\sec Desktops -- Groups % #############################################################################################
% ######################################################################################################################

Virtual desktops, sometimes called workspaces or window groups, offer a way of organizing applications.
Switching to different desktop hides all the applications from previous desktop and shows applications from new desktop.
This way, user can group applications that are used together and switch between different tasks easily.

Many slightly different models of virtual desktops exist among window managers.
The most standard model is using multiple desktops, one of them being visible at any given time
and window always belonging to exactly one of them.
In another model, more than one desktop could be visible at a time, or window can belong to more than one group, or both.

Firstly, lets briefly summarize what EWMH tells about virtual desktops~\cite[ewmh]:
\begitems
* only one desktop can be shown on the screen at a time,
* there may be a fixed number of desktops, or new ones may be created dynamically,
* window manager offers a way for the user to move clients between desktops,
* clients may be allowed to occupy more than one desktop simultaneously.
\enditems
For more detailed description, see section~\ref[sec:ewmhdesktops].

For SWM, we decided to build on cwm's desktop (group) model and design it in a similar way.
There is unlimited amount of standard groups (cwm has fixed amount of nine groups),
and  one special group (called sticky), that is always visible.
Every standard group could be either visible or hidden and each window can belong to arbitrary number of groups.
Window is visible if any of its group is visible, otherwise hidden.

Although this group model violates EWMH specification (only one desktop can be shown on the screen at a time),
it is the most versatile solution and, in a way, superset of all others.
Depending on the implementation (which will be discussed in section~\ref[sec:groupimpl]),
user can always opt to have only one group visible at a time and each window assigned to exactly one group.
