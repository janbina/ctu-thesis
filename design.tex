\label[chap:design]
\chap Design

% ######################################################################################################################
%\sec Design % ##########################################################################################################
% ######################################################################################################################

The primary design goal for swm was to be small, easy to use yet hackable stacking window manager.
To design such window manager, we based it on one of the core principles of Unix philosophy, that programs should do one thing well,
as stated by Doug McIlroy, one of the founders of the Unix tradition, in the Bell System Technical Journal~\cite[mcilroy]:
``Make each program do one thing well.
To do a new job, build afresh rather than complicate old programs by adding new features.''

To honor this principle, swm does not come with its own panel or application launcher, as some window managers do.
Instead, it focuses on that one thing, window management, and the rest is left for third party applications.
Thanks to ICCCM and EWMH specifications, third party applications can interact with swm seamlessly
and the user can choose any application that fits his needs instead of using the one bundled with the window manager.

For example, there are many open source panel implementations,
like polybar~\cite[polybar], lemonbar~\cite[lemonbar], or pypanel~\cite[pypanel], just to name a few.
To mention some application launchers that could be used with swm, we can go with dmenu~\cite[dmenu] or rofi~\cite[rofi].

In this chapter, we will discuss some of the design choices made during the development of swm into detail.

% ######################################################################################################################
\sec Requirements % ##################################################################################
% ######################################################################################################################

weweewdwed

% ######################################################################################################################
\sec Tools % ##########################################################################################################
% ######################################################################################################################

In this section, we will describe some tools that made certain design decisions possible.

% ######################################################################################################################
\label[sec:xdotool]
\secc Xdotool % ########################################################################################################
% ######################################################################################################################

Xdotool is a simple command line tool for communication with X server.
While its main use case is to fake input from the mouse and keyboard, it also supports some parts of EWMH protocol,
and thus lets you perform various window manager actions~\cite[xdotool].
Actions important for out usecase are:
\begitems
* "(get|set)_desktop" -- get and set the current desktop.
* "(get|set)_num_desktops" -- get and set number of desktops.
* "(get|set)_desktop_for_window" -- get and set the desktop for window.
* "windowactivate" -- activate a window.
Sends "_NET_ACTIVE_WINDOW", window manager should do necessary changes so that specified window could be shown,
for example switch to desktop the window is on, and then activate it (give it input focus).
* "windowminimize" -- minimize a window.
\enditems

To specify a window on which the action should be performed, one can either directly provide its ID, or
use one of those three commands to obtain it:

\begitems
* "getactivewindow" -- get an ID of active window.
* "selectwindow" -- get an ID of a window by clicking on it.
* "search" -- get IDs of windows whose name or class is matching the search term.
Various filters are available, you can, for example, limit the search only to windows located on the specified desktop.
\enditems

Xdotool can chain commands, so you can use one command to get a window ID first and then perform an action on it.
See listing~\ref[code:xdotool] for some examples of xdotool commands.

\codeblock{code:xdotool}{Xdotool commands}{xdotool.txt}

% ######################################################################################################################
\label[sec:wmctrl]
\secc Wmctrl % #########################################################################################################
% ######################################################################################################################

Wmctrl is, as well as xdotool, a command line tool for communication with X server.
Unlike xdotool, wmctrl is specialized directly on interaction with EWMH compatible window manager, though.
It is a bit more complex than xdotool, because, as stated in its documentation, it provides access to almost all the features defined in the EWMH specification~\cite[wmctrl].
This means that it can do all the window manager related tasks xdotool can do and much more, but the commands are usually not as straightforward.
That is also caused by arguments being named by a single letter, which mostly seems to be chosen by random.
We will show two commands which will be important for our use case, the rest could be seen in the documentation~\cite[wmctrl].

\begitems
* "wmctrl -c <WIN>" -- gracefully close specified window.
* "wmctrl -r <WIN> -b <STATE>" -- change the state of the specified window.
This command could be used, for example, to make the window maximized, minimized, or fullscreen.
It sends the "_NET_WM_STATE" client message as specified in EWMH.
The format of the state argument is: "(remove|add|toggle),<PROP1>[,<PROP2>]", and the following properties are supported:
\begitems \style x
* "modal",
* "sticky",
* "maximized_vert", "maximized_horz",
* "shaded",
* "skip_taskbar", "skip_pager",
* "hidden",
* "fullscreen",
* "above", "below".
\enditems
\enditems

The window, on which the action should be performed, could be specified in one of these ways:

\begitems
* By default, window argument is interpreted as a string matched against the window title and the first matching window is used.
* Using the "-i" option, the argument will be interpreted as a numerical window ID represented as a decimal or hexadecimal (prefix "0x") number.
* Special strings ":ACTIVE:" and ":SELECT:" may be used to use the currently active window or to let the user select the window by clicking on it, respectively.
\enditems

Wmctrl can be used to, for example, close the window or toggle its maximized state.
See listing~\ref[code:wmctrl] for some examples of wmctrl commands.

\codeblock{code:wmctrl}{Wmctrl commands}{wmctrl.txt}

% ######################################################################################################################
\label[sec:sxhkd]
\secc Sxhkd % ##########################################################################################################
% ######################################################################################################################

Sxhkd, standing for {\em Simple X hotkey daemon}, is an X daemon that reacts to input events by executing commands~\cite[sxhkd].
You provide it with one or more configuration files, which define the associations between the input events and the commands.

Example configuration file can be seen in code listing~\ref[code:sxhkd].
It demonstrates how sxhkd makes it very easy to map multiple shortcuts to multiple commands at once.
It is done using syntax "{_,A,B} + X", which will define three shortcuts -- "X", "A + X", and "B + X".
Underscore is representing an empty sequence element -- this is useful if we want to have different variants of command
based on modifiers used (as in the example, line 7).

\codeblock{code:sxhkd}{Example configuration file for sxhkd}{sxhkd.txt}

Sxhkd can execute defined commands both on key press (default behavior) and on key release (using modifier ``"@"'').
It has also very interesting feature called {\em chord chain} -- one can specify multiple chords separated by semicolons,
the command will then only be executed after receiving each chord of the chain in consecutive order.
One can also use colon instead of semicolon to indicate that the chord chain shall not be aborted when the chain tail is
reached -- see the example, line 4.
That means that if we take our example, upon pressing "super + alt + m", {\em moving mode} will be activated and we can
then move the window just by pressing keys "h, j, k, l".

% ######################################################################################################################
\sec Configuration and Controlling % ##################################################################################
% ######################################################################################################################

Most of the window managers are using configuration files with lots and lots of options you can configure.
With swm, we decided to take a different approach:
\begitems
* There is no need for swm to have commands for actions like maximizing or minimizing the window -- since it supports EWMH
and there are tools for sending EWMH messages (xdotool, wmctrl), we can just make use of them.
* There is no need to handle keyboard shortcuts -- users can use sxhkd that does it better than we could ever do.
\enditems
This means that to maximize and minimize the window, as well as to do many more actions, user can use sxhkd in combination
with xdotool or wmctrl and swm does not have to be involved at all (apart from implementing EWMH protocol).

Not everything could be done through EWMH and third party tools, though.
To address those cases, swm comes with two tools -- swmctl and swmrc -- that will be discussed in the next section.

% ######################################################################################################################
\sec Swmctl and Swmrc % ###############################################################################################
% ######################################################################################################################

Swmctl is a simple command line utility that can be used to send commands to swm.
It is primarily meant to be used in combination with sxhkd to invoke the commands with keyboard shortcuts,
but it can also be used directly from the command line.
This makes it easy to test the command first and edit the configuration file only after it works exactly how one wants it to.

All the swmctl's commands will be discussed one by one in chapter~\ref[chap:impl] later,
but to get an idea of how swmctl works and why it is better than simple configuration file, lets have a look at
following example.

To move a window using keyboard in cwm, one has to bind keyboard shortcuts in its configuration file like this:
\begtt
moveamount 1
bind CM-k  window-move-up
bind CMS-k window-move-up-big
\endtt
As you can see, there is an option "moveamount", which defines how many pixels the window will move.
Then, there are two commands for each direction "window-move-X" and "window-move-X-big".
The first one moves the window by "moveamount" pixels, the second one moves it by ten times the "moveamount" pixels.

In swmctl, on the other hand, there is a "move" command, that takes arguments for all the directions and amount of pixels to move the window by.
User can then bind this command to keyboard shortcut using sxhkd like this:
\begtt
ctrl + alt + {h,j,k,l}
    swmctl move -{w,s,n,e} 20
\endtt

This approach is both much more simple and versatile than cwm's:
\begitems
* while cwm is limited to two move amounts (and one is even dependant on the other), swm can handle arbitrary move amount,
* swm can move the window in multiple directions in one command, and it can even be moved by different amount of pixels in each direction.
\enditems

Swmrc is just a script that is executed upon startup of swm.
It is executed after all the initialization has been done, so the window manager is running and
can accept any command (swmctl's one, EWMH message sent using xdotool), but right before all the existing windows
are adopted by swm.
This makes it ideal for all the configuration commands -- one can, for example, set the names of the groups, color of borders etc.

% ######################################################################################################################
\sec Desktops -- Groups % #############################################################################################
% ######################################################################################################################

Virtual desktops, sometimes called workspaces or window groups, offer a way of organizing applications.
Switching to different desktop hides all the applications from the previous desktop and shows applications from a new desktop.
This way, the user can group applications that are used together and switch between different tasks easily.

Many slightly different models of virtual desktops exist among window managers.
The most standard model is using multiple desktops, one of them being visible at any given time
and window always belonging to exactly one of them.
In other models, more than one desktop could be visible at a time and a window can belong to more than one group.

Firstly, lets briefly summarize what EWMH tells us about virtual desktops~\cite[ewmh]:
\begitems
* only one desktop can be shown on the screen at a time,
* there may be a fixed number of desktops, or new ones may be created dynamically,
* window manager offers a way for the user to move clients between desktops,
* clients may be allowed to occupy more than one desktop simultaneously.
\enditems
For more detailed description, see section~\ref[sec:ewmhdesktops].

For swm, we decided to build on cwm's group model and design it in a similar way.
There is an unlimited amount of standard groups (cwm has a fixed amount of nine groups),
and one special group (called sticky), that is always visible.
Every standard group could be either visible or hidden and each window can belong to an arbitrary number of groups.
A window is visible if any of its group is visible, otherwise hidden.

Although this group model violates EWMH specification (``only one desktop can be shown on the screen at a time''),
it is the most versatile solution and, in a way, a superset of all others.
Depending on the implementation (which will be discussed in section~\ref[sec:groupimpl]),
the user can always opt to have only one group visible at a time and each window assigned to exactly one group.
