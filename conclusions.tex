% ######################################################################################################################
\label[chap:conclusions]
\chap Conclusion
% ######################################################################################################################

In this thesis, we described the working mechanisms of the X Window System Protocol
and two standard protocols defined on top of it, ICCCM and EWMH.
Those mechanisms and policies were described especially concerning window managers,
special X Server clients that control the placement and appearance of windows on the screen.
We also described what a window manager is, its different types,
and features it usually offers, demonstrated on some existing implementations.
Finally, a new stacking window manager called swm was designed in such a way
that it is small, extensible, and scriptable.
It was implemented in Go, a modern system programming language.
It is ICCCM and EWMH compliant, and its core features include stacking window management,
advanced window grouping mechanism, and basic window decorations.
It is extensible and scriptable and comes with a command-line
utility that makes it easy to manipulate windows in a simple but powerful way.
We also described how a window manager could be tested and how tests for core features of swm were implemented.

% ######################################################################################################################
\nonum
\sec Future Work
% ######################################################################################################################

Swm is an open-source project, its source code is published on GitHub\urlnote{https://github.com/janbina/swm}
under the MIT license.
It was designed and implemented in a way to be built upon, to be extended with new functionalities,
and adapted to various use cases and user preferences.
Therefore, there is a potential for it to be extended by third parties,
either in a way of contributions or by starting a new project using swm as a starting point.
