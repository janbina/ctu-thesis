\chap Implementation

% ======================================================================================================================
\label[sec:xlibAndXCB]  % ==============================================================================================
\sec Xlib and XCB % ====================================================================================================
% ======================================================================================================================

%Xlib and XCB are two helper libraries that provide API for talking to the X server.
%
%Xlib, also known as libX11, is the original C language X11 API, released in 1985.
%It was designed to look like a traditional library API, hiding the fact that calls result
%in protocol requests to a server.
%Calls that don't require a response from the X server are queued in a buffer
%to be sent as a batch of requests to the server.
%Those that require a response flush all the buffered requests and then
%block until the response is received.
%This mix of synchronous and asynchronous behavior causes some problems,
%because it is not obvious which calls implicitly flush the buffer and which
%do not.\cite[XlibXCB]
%
%XCB is a second attempt at defining a C language binding for X11.
%It was first released in 2001, after many years of experience with Xlib,
%learning from it and other protocol interface libraries.
%XCB makes the client-server nature of the protocol explicit in its design.
%The client is in charge of deciding when to flush the request buffer,
%when to read results and when to wait for the server to respond.\cite[XlibXCB]
%
%In the code listings~\ref[code:xlib] and~\ref[code:xcb], we can see comparison of those two libraries on the task of looking up window property.
%Xlib generates the request to the X server to retrieve the property and appends it to its buffer of requests.
%Since this is a request that requires a response, Xlib then flushes the buffer, sending the contents to the X server.
%Next, Xlib waits until the X server processes all the requests preceding the property retrieve request, and sends the property retrieve reply.
%Xlib then returns the reply to the client.
%XCB on the other hand, provides functions generated directly from the protocol descriptions in an "obvious" mechanistic way.
%XCB functions map directly onto the protocol, with separate functions to put requests into the outgoing buffer and to read results back from the X server asynchronously later.
%The power of XCB is in allowing those two steps to have as much code as you want between them.
%The programmer decides when to wait for data, instead of being forced to wait for the data returned by a request at the time the request is issued.\cite[XlibXCB]
%
%\codeblock{code:xlib}{Window property lookup using Xlib}{winproperty_xlib.c}
%\codeblock{code:xcb}{Window property lookup using XCB}{winproperty_xcb.c}
%
%Xlib and XCB are compatible, meaning that one can mix calls to the first with calls to the other.
%This compatibility was achieved by rebuilding libX11 as a layer on top of libxcb.
%Xlib and XCB share the same X server connection and pass control of it back and forth.
%That option was introduced in libX11 version 1.2, and since version 1.4, released in 2010, it is always present.\cite[XlibXCB]
%
%Most applications should call Xlib and XCB sparingly.
%Higher level toolkits provide more efficient programming models, and support features expected in modern applications,
%including support for complex internationalized input and output, accessibility, and integration with desktop environments.
%However, sometimes applications will find themselves needing to make calls to the raw underlying X11 libraries for operations not supported by toolkits.
%For example, an application might need to make calls to X extension API's not covered in the current version of the toolkit's programming model.\cite[XlibXCB]

% ======================================================================================================================
\sec Go % ==============================================================================================================
% ======================================================================================================================

% ======================================================================================================================
\sec XGB and xgbutil % =================================================================================================
% ======================================================================================================================

There are two unofficial libraries for accessing X11 API from Go,
xgb\urlnote{https://github.com/BurntSushi/xgb} and xgbutil\urlnote{https://github.com/BurntSushi/xgbutil}.

XGB is closely modeled after XCB, so it is just a low-level API to communicate with the
core X protocol and many of the X extensions.
It claims being thread safe and according to benchmarks, it gets immediate improvement from parallelism.
In the code listing~\ref[code:xgb], you can see example of window property lookup using XGB.
It is very similar to XCB version~\ref[code:xcb], the only difference being that we can simply
call "Reply()" method on the cookie object returned from "GetProperty" function, instead of passing the cookie to
another function.

\codeblock{code:xgb}{Window property lookup using XGB}{winproperty_go.go}

Xgbutil, on the other hand, is higher level utility library working on top of the XGB.
Its main goal is to make various X related tasks easier.
Those might be, for example:
\begitems
* binding keys,
* using the EWMH or ICCCM specs with the window manager,
* moving and resizing windows,
* assigning function callbacks to particular events, and others.
\enditems

To get an idea about the design of xgbutil, we can have a look at some functions in the code listing~\ref[code:xgbutil].
Functions like "icccm.WmNormalHintsGet" could be used to get specific window property ("WM_NORMAL_HINTS" in this example).
Internally, it calls "xprop.GetProperty", which is a wrapper for "xproto.GetProperty" from XGB.
It has no cookie/reply mechanism though, so we are losing the asynchronicity of XCB here.
We can, however, always fall back to using XGB for cases in which we need to be asynchronous.
To process raw "GetPropertyReply", which itself is just a byte array with some meta data,
xgbutil defines set of functions like "PropValNums()" and "PropValWindows()".
These extract slice of integers and slice of window identifiers, respectively, out of the "GetPropertyReply".

\codeblock{code:xgbutil}{Xgbutil API showcase}{xgbutil.go}
